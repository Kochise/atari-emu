               ADDITIONAL INSTRUCTIONS FOR USING OMEN

Here are some more detailed basic instructions for using the OMEninterface, especially for opening and using programs and files. Thetopics covered are listed below.

*** Note: SOURCE CODE SAMPLE ***
*** SCREEN SCROLL ***
*** VIEWING FILES ***
*** VIEWING OMEN SYSTEM FILES ***
*** DIRECTORY SCROLL CONTROL ***
*** VIEWING DEMO DISK FILES ***
*** OPENING (RUNNING) SOFTWARE ***
*** OPENING DOCUMENT FILES ***
*** COPYING THE DEMO DISK ***
*** VISIT OMEN's HELP" MENUS ETCETERA ***


*** Note: SOURCE CODE SAMPLE ***

(This note has been placed here because it was omitted in theoriginal release notes. It is not related to the instructions thatfollow.)

Programmers should note the sample Eazy`Asm source code for theStereo`Sound`Rate`Reducer" program. It cuts the sample rate andhence the size of stereo digitized sounds in half, with some loss offidelity.

This utility program was written in under two hours. A bug foundlater was fixed and the new version saved in under five minutes. (Afew of the first demo disks went out with the bugged version, whichmade the new file size 140 bytes too short.)


*** SCREEN SCROLL ***

Atari based OMEn uses a display which is twice as tall as the actualscreen, except in TT Hirez where the screen is already two pages. Inorder to scroll down to see the bottom half of the display, simplymove the mouse to the bottom of the screen and keep going - thedisplay will scroll. To scroll back to the top, just move to the topand keep going in the same manner as going down.

The screen scroll is done with Atari's scrolling hardware. As far assoftware is concerned, the screen is actually twice as tall as whatis seen.


*** VIEWING FILES ***

To view files, a DIRectory window must be opened, by clicking onDIRs" at the top of the display. Don't move the mouse at all whenclicking, otherwise the window won't open.

Files are shown with their name followed either by a picture (icon")(for common types of files) or a four character file extension. Hereare the uses of the icons:
   Directory     Folder Icon
   Directory     Binder Icon (Binder is a special type of directory)
   Text File     Piece of Paper
   Program       Running animal (you can run" the program)
   I/O Protocol  Diamond
   I/O Port      Double Arrow
   I/O Channel mgr.  Diamond with double arrow inside
   Sound         Speaker

Initially, the directory window opens showing the Memory" rootdirectory. In the memory directory are two subdirectories: Disk`Drives" and System".


*** VIEWING OMEN SYSTEM FILES ***

To view the OMEn system files, first open a directory window (seeViewing Files, above). Click on System" without moving the mouse.The display in the window will change to show the files inside theSystem" directory, and the word System" will replace Memory" onthe second line of the window.

All OMEn's Input/ Output manager files are in System", and there arealso some application programs there. Files in System" are alreadyloaded when OMEn is started, so they will start without accessing thedisk drive.

Clicking on a program will start it running - see Opening Software",below.

To leave the System" directory, click on the word System" up nearthe top of the directory window. This will exit from the currentdirectory (System) and go out to the previous directory (Memory).


*** DIRECTORY SCROLL CONTROL ***

When there are more files in a directory than will fit in thedirectory window, the Scroll Bar along the left side of the windowbecomes active. To go up or down by a windowfull of files at a time,click on the up or down arrow keys without moving the mouse. To go upor down by just one file, click on the up or down carat symbols. Togo directly to any area of the window, click on the approximateposition desired in the vertical scroll bar underneath these symbols.Unlike GEM, the scroll bar will jump directly to that position. To dolive scrolling", hold the mouse button down in the scroll bar areaand move the mouse slowly up and down the scroll bar.
*** VIEWING DEMO DISK FILES ***

To view files on the demo disk, go back to the Memory" directory andclick on Disk`Drives. The display will change to show all the GEMdrives on your system.

Click on Drive`A (without moving the mouse). The list of the files inthe main (root") directory of the disk will be shown. At this pointthe name Disk`Drives" will be at the very top of the directorywindow, and Drive`A" will be underneath it.

Of course, a directory can be opened for any drive on the system(floppy disk B and hard drives) by clicking on that drive.

Note: The file names on disks depend on whether the file was copiedusing GEM or OMEn. OMEn uses 14 character upper and lower case filenames on GEM disks, while GEM itself uses 8 character upper case filenames. Copying an OMEn file with GEM will cause the file name to bechanged to the GEM format. To archive OMEn for BBS downloading, itwas necessary to use GEM based software, which shortened the filenames. Regrettably, the remaining 8 characters don't always describethe file very well, such as OMEn`User`Docs" being shortened toOMEN`USE".

In the main directory of the demo disk, there are several subdir-ectories shown. To find the application programs on the disk, clickon the Software" (or SOFTWARE") directory. To find the document-ation, click on OMEn`User`Docs" (or OMEN`USE") and so on. There areadditional directories for Fonts, Pictures and Sounds. The Settings"directory contains only a text file describing settings.


*** OPENING (RUNNING) SOFTWARE ***

To open (run) an application program, you must locate it (above) andthen click (without moving the mouse) on its name in the directory.The program will load and its window will open without affecting therest of the display.

By opening Text`Ed, you can view the instructions for all theindividual progams supplied. (See Opening Document Files, below).

Don't make the mistake of trying to run a program by clicking on itsinstructions in E_Util`Progs" (or E_UTILPR") documentation direc-tory instead of the program itself, which is in the Software"directory. Application programs have a running animal" icon; theOMEn text documents have a piece of paper" icon. Directories have afolder" icon.
*** OPENING DOCUMENT FILES ***

Once an application program is running, it can open documents of itsproper type. It is necessary to exit the Software" or System"directory where the program was found and go the directory where thedocuments files are to be found (See Viewing Demo DiskFiles, above).

Or, the Software" directory can be left open and the documentdirectory viewed in a second directory window, opened by clickingagain on DIRs.

When the document to be opened is located, press down the mousebutton on the file and, with the button still pressed down, move themouse into the window of the program that is to open the file. Thenlet go of the mouse button, and the file will be opened by theprogram. (With Text`Ed and text files, it is easiest to drag the fileto Text`Ed's File" button.)

Opening" a file can mean different things to different programs.Play`It`Again`Sam will play digitized sound files, while Font`Demowill make a bitmapped font for display or printout from an outlinefont definition file.

Here is a list of document file types for the demo programs. Theprograms not shown do not open document files of any type.

   PROGRAM      OPENS              FILE EXTENSION   ICON

   Text`Ed      OMEn Text files    TEXT             Piece of paper
   Paint`Booth  Bitmapped pictures BM:BMAP,BM01,... None
   Sound Player (Play`It`Again`Sam")
                Digitized Sounds   SOUN             Speaker
   Font`Demo    Outline Fonts      PS3F,OM3F,PS     None
   File`Info    Any file
   Binary`Ed    Any file (This is for programmers)
   Cheap`OMEN`TO`ASCII
   - Same as Text`Ed (changes it to TXT" for GEM)
   Stereo`Sound`Rate`Reducer
         Stereo Digitized Sounds   SOUN             Speaker


*** COPYING THE DEMO DISK ***

If the demo disk is copied from GEM, the file names are shortened toeight upper case characters and the file extensions are shortened tothree characters. Because of this, the Text`Ed icons don't work andtwo of them become duplicate file names. In addition, and moreserious, the copy of Text`Ed itself doesn't work for reasons yet tobe investigated.

The best way to copy the entire demo from GEM is to format another720K disk and do a complete disk copy by dragging the drive A icononto the drive B icon. Alternatively, a disk copy program can be usedto copy all the sectors of the disk.

From OMEn, without a hard drive, one floppy, and more than onemegabyte of memory, drag the Drive`A" directory file into theMemory" folder.
Place the copy disk in the drive. Enter the original Drive`Afolder in Disk`Drives" to access the new disk. Open the copyDrive`A" in Memory", then click on Select All" from the Dir`Misc"menu.
Drag the selected files to the directory of the new disk.
Unfortunately, the current version of OMEn is quite slow withfloppy disks.

With one megabyte or less of memory and a single floppy disk, theentire disk can't be copied at once as there isn't enough memory. (Aone meg ST with no desk accessories open holds the entire 719K demodisk, and then runs out of memory trying to open the new blankdisk--rats!) The files must be copied into Memory" in groups,transferred to the new disk, and then erased from memory to free upspace. Drive`A must be closed and re-opened each time the disk isswapped. It's easier to do the Disk Copy" from GEM.

With two floppy drives, coping the demo to a disk in Drive`B iseasier. Enter the Drive`A folder, viewing the original disk. Open asecond directory window and enter the Drive`B folder, viewing theblank disk. In Drive`A, choose Select All" from the File Misc"menu, then drag the selected files to Drive`B".

To copy the demo to a hard drive under OMEn is easiest.
Open a second directory window, and enter Drive`C". Drag theDrive`A" folder into it from the first directory window. Rename thecopy of Drive`A" (which is now on Drive`C) to OMEN" to complete theinstallation.


*** VISIT OMEN's HELP" MENUS ETCETERA ***

The instuctions in here can help get you started. There are moreinstructions in the 2.46 Release Notes which can be printed from GEM,in the Help" menus at the top of the OMEn display, and in thedocumentation files in the User`Docs directory on the demo disk. Ingeneral, the User`Docs cover the more advanced topics, while Help"and the release notes are more basic.
